% Chapter 4

\chapter{Conclusioni} % Main chapter title

\label{Conclusioni} % For referencing the chapter elsewhere, use \ref{Chapter1} 

\lhead{\emph{Conclusioni}} % This is for the header on each page - perhaps a shortened title

%----------------------------------------------------------------------------------------

I risultati ottenuti dalle simulazioni effettuate mediante
l'utilizzo dell'algoritmo implementato in CUDA, messi a confronto con i
risultati del medesimo algoritmo codificato in Python, hanno confermato
la correttezza dell'algoritmo CUDA. Differentemente dalla versione Python, che fa uso di una
struttura a stack più intuitiva da implementare e gestire,
l'algoritmo CUDA utilizza una struttura dati ad albero
per l'elaborazione della simulazione. La scelta di questo tipo di strutture 
ad albero ha portato un enorme
vantaggio in termini prestazionali, poiché: la scelta di implementare
una soluzione che prevedesse l'utilizzo del parallelismo dinamico ha permesso di
verificare la validità di questa metodologia nella simulazione di fenomeni
intrinsecamente ricorsivi. Il parallelismo dinamico con molti livelli
di profondità risulta quindi un'ottima soluzione quando si tratta di elaborare
strutture dati predisposte alla ricorsione, come in questo caso. Sebbene CUDA
preveda un limite fisico al numero di livelli di ricorsione disponibili, è
degna di nota l'accelerazione fornita alla simulazione.
Gli sviluppi futuri prevedono, a fronte della crescita esponenziale della 
popolazione cellulare, l'ottimizzazione dell'utilizzo della memoria 
tramite l'implementazione di tecniche per la gestione di un numero 
di GPU superiore ad uno. Così facendo, sarebbe possibile demandare la computazione 
di sottopopolazioni cellulari ognuna ad una GPU dedicata, incrementando 
ulteriormente le performance del simulatore. Inoltre, sarebbe interessante 
analizzae lo speedup a fronte di una popolazione iniziale $X_{0}$ con 
numerosità estremamente maggiore di quella utilizzata per le simulazioni 
riportate nella Sezione \ref{sec:performance}, approssimativamente di $5*10^4$
cellule.


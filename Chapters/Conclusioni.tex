% Chapter 4

\chapter{Conclusioni} % Main chapter title

\label{Conclusioni} % For referencing the chapter elsewhere, use \ref{Chapter1} 

\lhead{\emph{Conclusioni}} % This is for the header on each page - perhaps a shortened title

%----------------------------------------------------------------------------------------

I risultati ottenuti dalle simulazioni effettuate mediante
l'utilizzo dell'algoritmo implementato in CUDA messi a confronto con i
risultati del medesimo algoritmo codificato in Python hanno confermato
la correttezza dell'algoritmo CUDA, dato che utilizza una struttura dati
ad albero per l'elaborazione della simulazione, differentemente dalla versione
Python che fa uso di una struttura a stack, più intuitiva da implementare e
gestire. La scelta di questo tipo di strutture ad albero ha portato un enorme
vantaggio in termini prestazionali, poiché è stato scelto di implementare
una soluzione che prevedesse l'utilizzo del parallelismo dinamico per
verificare la validità di questa metodologia nella simulazione di fenomeni
intrinsecamente ricorsivi. Il parallelismo dinamico con molti livelli
di profondità risulta quindi un'ottima soluzione quando si tratta di elaborare
strutture dati predisposte alla ricorsione, come in questo caso. Sebbene CUDA
preveda un limite fisico al numero di livelli di ricorsione disponibili, è
degna di nota l'accelerazione fornita alla simulazione.
L'aumento delle performance è evidente anche utilizzando i dati di test
forniti, ma sarebbe interessante vedere a confronto il comportamento
dell'algoritmo parallelo con quello sequenziale a fronte di una popolazione
iniziale $X_{0}$ con numerosità di gran lunga più elevata, e a fronte
di un tempo massimo di proliferazione $\tau_{max}$ molto distante.
Gli sviluppi futuri prevedono l'ottimizzazione dell'utilizzo della memoria
a fronte della crescita esponenziale della popolazione cellulare.

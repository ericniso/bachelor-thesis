% Chapter 1

\chapter{Introduzione} % Main chapter title

\label{Introduzione} % For referencing the chapter elsewhere, use \ref{Chapter1} 

\lhead{\emph{Introduzione}} % This is for the header on each page - perhaps a shortened title

%----------------------------------------------------------------------------------------

L’evoluzione di popolazioni di cellule proliferanti risulta spesso estremamente
difficile da caratterizzare mediante convenzionali tecniche di biologia
sperimentale. Tuttavia, un’analisi approfondita di questi fenomeni
faciliterebbe lo studio e la comprensione di diverse patologie umane.
Per questa ragione, è necessario accoppiare metodologie computazionali 
avanzate ai classici esperimenti di laboratorio. Nel caso particolare della 
Leucemia mieloide acuta (LMA) si pensa che l'eterogeneità tra diverse 
sottopopolazioni cellulari presenti all'interno del tumore giochi un ruolo 
fondamentale per quanto riguarda la resistenza alle cure e il fatto che il 
tumore possa ripresentarsi a seguito del completamento della terapia.
Per lo studio di questo fenomeno si può optare per un approccio model-driven 
in modo da studiare le diverse dinamiche riguardanti la proliferazione di 
popolazioni cellulari. L'obiettivo di questo stage è lo sviluppo di un 
simulatore di divisione cellulare in grado di tenere conto della presenza 
di differenti tipi di popolazioni cellulari e della stocasticità degli eventi
di divisione ad essi correlati. Utilizzando un simulatore software implementato 
solamente tramite sviluppo su CPU è emerso l'elevato costo 
computazionale necessario alla simulazione di eventi di divisione riguardanti 
popolazioni cellulari di grandi dimensioni, a causa di ciò si è reso necessario 
l'utilizzo di tecniche di accelerazione software per incrementare le 
performance del simulatore. È stato scelto di sviluppare il software 
mediante la libreria CUDA che dà la possibilità di sfruttare il parallelismo 
offerto dalle Graphics Processing Units moderne per ottenere incrementi 
significativi delle performance delle simulazioni effettuate su modelli di 
proliferazione cellulare riguardanti la leucemia mieloide acuta.
Durante lo sviluppo del simulatore è stato implementato l'utilizzo del 
parallelismo dinamico offerto dalla libreria CUDA per simulare questi fenomeni 
di divisione cellulare inerentemente ricorsivi, trovando immediato riscontro 
positivo verificabile grazie al significativo incremento delle performance 
del simulatore. Il confronto dei dati estrapolati dalle simulazioni effettuate
tramite simulatore per CPU e la corrispondente versione parallela 
sviluppata durante il periodo di stage hanno confermato la correttezza dei 
risultati ottenibili utilizzando la versione del software implementata con
l'ausilio della libreria CUDA.

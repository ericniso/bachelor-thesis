\section{Confronto con dati sperimentali}

Lo scopo della simulazione è verificare la correttezza del modello utilizzato
sulla base di dati reali provenienti da esperimenti effettuati in vivo
in laboratorio.
\\
In Figura \ref{chart:comparison} è possibile vedere la comparazione tramite 
overlap dei
risultati ottenuti dalle simulazioni fra l'algoritmo di riferimento sviluppato
in Python e la versione implementata in CUDA.
Nei grafici sono presenti aree in cui
i valori di frequenza si discostano tra loro di alcune unità. Questa variazione
è dovuta al fatto che la simulazione deve elaborare eventi di tipo stocastico,
dunque il numero di cellule aventi un determinato tipo e il timer di divisione
non saranno mai esattamente uguali per ogni simulazione che si andrà ad
effettuare, e questa situazione è appunto evidenziata dalle aree del grafico
dove è assente l'overlap dei valori. 

\begin{figure}[H]
    \begin{minipage}[b]{.5\linewidth}
        \centering
        \scalebox{0.65}{
        \pgfplotsset{scaled y ticks=false}
        \begin{tikzpicture}
            \begin{axis}[
                xmode=log,
                area style,
                ylabel={frequenza cellule},
                xlabel={valori di fluorescenza}
                ]
            \addplot+[ybar interval,mark=no]
                table {Data/Final/fit_cuda.txt};
            \addplot+[ybar interval,mark=no,opacity=0.5]
                table {Data/Final/fit_cpu.txt};
            \legend{CUDA, Python}
        \end{axis}
        \end{tikzpicture}
        }
        \subcaption{Confronto simulazioni Python e CUDA con
        \\
        $\varphi_{min}=11.0, \tau_{max}=240$}
    \end{minipage}
    \begin{minipage}[b]{.5\linewidth}
        \centering
        \scalebox{0.65}{
        \pgfplotsset{scaled y ticks=false}
        \begin{tikzpicture}
            \begin{axis}[
                xmode=log,
                area style,
                ylabel={frequenza cellule},
                xlabel={valori di fluorescenza}
                ]
            \addplot+[ybar interval,mark=no]
                table {Data/Final/validation_cuda.txt};
            \addplot+[ybar interval,mark=no,opacity=0.5]
                table {Data/Final/validation_cpu.txt};
                \legend{CUDA, Python}
        \end{axis}
        \end{tikzpicture}
        }
        \subcaption{Confronto simulazioni Python e CUDA con
        \\
        $\varphi_{min}=8.7, \tau_{max}=504$}
    \end{minipage}
    \begin{minipage}[b]{.5\linewidth}
        \vspace{0.3cm}
        \centering
        \scalebox{0.65}{
        \pgfplotsset{scaled y ticks=false}
        \begin{tikzpicture}
            \begin{axis}[
                xmode=log,
                area style,
                ylabel={frequenza cellule},
                xlabel={valori di fluorescenza}
                ]
            \addplot+[ybar interval,mark=no]
                table {Data/Final/fit_cuda.txt};
            \addplot+[ybar interval,mark=no,opacity=0.5]
                table {Data/Final/fit_target.txt};
            \legend{CUDA, Laboratorio}
        \end{axis}
        \end{tikzpicture}
        }
        \subcaption{Confronto simulazione CUDA con
        \\
        $\varphi_{min}=11.0, \tau_{max}=240$
        \\
        e dati sperimentali}
    \end{minipage}
    \begin{minipage}[b]{.5\linewidth}
        \vspace{0.3cm}
        \centering
        \scalebox{0.65}{
        \pgfplotsset{scaled y ticks=false}
        \begin{tikzpicture}
            \begin{axis}[
                xmode=log,
                area style,
                ylabel={frequenza cellule},
                xlabel={valori di fluorescenza}
                ]
            \addplot+[ybar interval,mark=no]
                table {Data/Final/validation_cuda.txt};
            \addplot+[ybar interval,mark=no,opacity=0.5]
                table {Data/Final/validation_target.txt};
            \legend{CUDA, Laboratorio}
        \end{axis}
        \end{tikzpicture}
        }
        \subcaption{Confronto simulazione CUDA con
        \\
        $\varphi_{min}=8.7, \tau_{max}=504$
        \\
        e dati sperimentali}
    \end{minipage}
    \caption{Confronto dei valori di fluorescenza sperimentali ricavati in laboratorio e
    quelli ottenuti dalle simulazioni utilizzando
    l'algoritmo sviluppato in Python e in CUDA con diversi parametri
    iniziali}
    \label{chart:comparison}
\end{figure}
